% Options for packages loaded elsewhere
% Options for packages loaded elsewhere
\PassOptionsToPackage{unicode}{hyperref}
\PassOptionsToPackage{hyphens}{url}
\PassOptionsToPackage{dvipsnames,svgnames,x11names}{xcolor}
%
\documentclass[
  american,
  11pt,
  11pt,
  letterpaper,
  onecolumn]{article}
\usepackage{xcolor}
\usepackage[top=1.2in,bottom=1.2in,left=1.25in,right=1.25in]{geometry}
\usepackage{amsmath,amssymb}
\setcounter{secnumdepth}{5}
\usepackage{iftex}
\ifPDFTeX
  \usepackage[T1]{fontenc}
  \usepackage[utf8]{inputenc}
  \usepackage{textcomp} % provide euro and other symbols
\else % if luatex or xetex
  \usepackage{unicode-math} % this also loads fontspec
  \defaultfontfeatures{Scale=MatchLowercase}
  \defaultfontfeatures[\rmfamily]{Ligatures=TeX,Scale=1}
\fi
\usepackage{lmodern}
\ifPDFTeX\else
  % xetex/luatex font selection
\fi
% Use upquote if available, for straight quotes in verbatim environments
\IfFileExists{upquote.sty}{\usepackage{upquote}}{}
\IfFileExists{microtype.sty}{% use microtype if available
  \usepackage[]{microtype}
  \UseMicrotypeSet[protrusion]{basicmath} % disable protrusion for tt fonts
}{}
\usepackage{setspace}
\makeatletter
\@ifundefined{KOMAClassName}{% if non-KOMA class
  \IfFileExists{parskip.sty}{%
    \usepackage{parskip}
  }{% else
    \setlength{\parindent}{0pt}
    \setlength{\parskip}{6pt plus 2pt minus 1pt}}
}{% if KOMA class
  \KOMAoptions{parskip=half}}
\makeatother
% Make \paragraph and \subparagraph free-standing
\makeatletter
\ifx\paragraph\undefined\else
  \let\oldparagraph\paragraph
  \renewcommand{\paragraph}{
    \@ifstar
      \xxxParagraphStar
      \xxxParagraphNoStar
  }
  \newcommand{\xxxParagraphStar}[1]{\oldparagraph*{#1}\mbox{}}
  \newcommand{\xxxParagraphNoStar}[1]{\oldparagraph{#1}\mbox{}}
\fi
\ifx\subparagraph\undefined\else
  \let\oldsubparagraph\subparagraph
  \renewcommand{\subparagraph}{
    \@ifstar
      \xxxSubParagraphStar
      \xxxSubParagraphNoStar
  }
  \newcommand{\xxxSubParagraphStar}[1]{\oldsubparagraph*{#1}\mbox{}}
  \newcommand{\xxxSubParagraphNoStar}[1]{\oldsubparagraph{#1}\mbox{}}
\fi
\makeatother


\usepackage{longtable,booktabs,array}
\usepackage{calc} % for calculating minipage widths
% Correct order of tables after \paragraph or \subparagraph
\usepackage{etoolbox}
\makeatletter
\patchcmd\longtable{\par}{\if@noskipsec\mbox{}\fi\par}{}{}
\makeatother
% Allow footnotes in longtable head/foot
\IfFileExists{footnotehyper.sty}{\usepackage{footnotehyper}}{\usepackage{footnote}}
\makesavenoteenv{longtable}
\usepackage{graphicx}
\makeatletter
\newsavebox\pandoc@box
\newcommand*\pandocbounded[1]{% scales image to fit in text height/width
  \sbox\pandoc@box{#1}%
  \Gscale@div\@tempa{\textheight}{\dimexpr\ht\pandoc@box+\dp\pandoc@box\relax}%
  \Gscale@div\@tempb{\linewidth}{\wd\pandoc@box}%
  \ifdim\@tempb\p@<\@tempa\p@\let\@tempa\@tempb\fi% select the smaller of both
  \ifdim\@tempa\p@<\p@\scalebox{\@tempa}{\usebox\pandoc@box}%
  \else\usebox{\pandoc@box}%
  \fi%
}
% Set default figure placement to htbp
\def\fps@figure{htbp}
\makeatother



\ifLuaTeX
\usepackage[bidi=basic]{babel}
\else
\usepackage[bidi=default]{babel}
\fi
% get rid of language-specific shorthands (see #6817):
\let\LanguageShortHands\languageshorthands
\def\languageshorthands#1{}
\ifLuaTeX
  \usepackage[english]{selnolig} % disable illegal ligatures
\fi


\setlength{\emergencystretch}{3em} % prevent overfull lines

\providecommand{\tightlist}{%
  \setlength{\itemsep}{0pt}\setlength{\parskip}{0pt}}



 
\usepackage[style=apa,backend=biber,style=apa,natbib=true]{biblatex}
\addbibresource{References.bib}


% Font and encoding setup for XeLaTeX
\usepackage{fontspec}
\usepackage{unicode-math}

% Math packages (order matters for XeLaTeX)
\usepackage{amsmath}
\usepackage{mathtools}

% Define mathbbm for indicator functions
\newcommand{\mathbbm}[1]{\mathbb{#1}}

% Bold math symbols
\let\boldsymbol\symbf

% Table and layout packages
\usepackage{booktabs}
\usepackage{longtable}
\usepackage{array}
\usepackage{multirow}
\usepackage{wrapfig}
\usepackage{float}
\usepackage{colortbl}
\usepackage{pdflscape}
\usepackage{tabu}
\usepackage{threeparttable}
\usepackage{threeparttablex}
\usepackage[normalem]{ulem}
\usepackage{makecell}
\usepackage{xcolor}

% Typography and formatting
\usepackage{microtype}
\usepackage{setspace}
\usepackage{fancyhdr}
\usepackage{titlesec}
\usepackage{caption}

% Set main font
\setmainfont{Times New Roman}[
  Ligatures=TeX,
  Numbers=OldStyle
]

% Set math font
\setmathfont{Latin Modern Math}

% Table and figure captions
\captionsetup[table]{skip=12pt, font=small, labelfont=bf}
\captionsetup[figure]{skip=12pt, font=small, labelfont=bf}

% Professional section formatting
\titleformat{\section}{\large\bfseries\sffamily\color{NavyBlue}}{\thesection}{1em}{}
\titlespacing*{\section}{0pt}{24pt}{12pt}

\titleformat{\subsection}{\normalsize\bfseries\sffamily\color{NavyBlue}}{\thesubsection}{1em}{}
\titlespacing*{\subsection}{0pt}{18pt}{9pt}

\titleformat{\subsubsection}{\normalsize\bfseries\sffamily\color{NavyBlue}}{\thesubsubsection}{1em}{}
\titlespacing*{\subsubsection}{0pt}{12pt}{6pt}

% Custom colors matching your theme
\definecolor{sunnyyellow}{RGB}{255, 223, 0}
\definecolor{sunnyblue}{RGB}{30, 144, 255}
\definecolor{sunnygray}{RGB}{248, 249, 250}

% Professional header and footer
\pagestyle{fancy}
\fancyhf{}
\rhead{\small\thepage}
\lhead{\small\textit{The Antares Mathematical Framework}}
\renewcommand{\headrulewidth}{0.5pt}
\renewcommand{\headrule}{\hbox to\headwidth{\color{NavyBlue}\leaders\hrule height \headrulewidth\hfill}}
\setlength{\headheight}{14pt}

% Better typography and spacing
\setlength{\emergencystretch}{3em}
\tolerance=9999
\hbadness=10000
\raggedbottom

% Paragraph spacing
\setlength{\parskip}{8pt}
\setlength{\parindent}{0pt}

% Better table spacing
\renewcommand{\arraystretch}{1.2}

% Abstract formatting
\renewenvironment{abstract}
  {\small\quotation\noindent\rule{\linewidth}{.5pt}\par\smallskip
   \noindent\textbf{Abstract.}\space}
  {\par\smallskip\noindent\rule{\linewidth}{.5pt}\endquotation}
\makeatletter
\@ifpackageloaded{caption}{}{\usepackage{caption}}
\AtBeginDocument{%
\ifdefined\contentsname
  \renewcommand*\contentsname{Table of contents}
\else
  \newcommand\contentsname{Table of contents}
\fi
\ifdefined\listfigurename
  \renewcommand*\listfigurename{List of Figures}
\else
  \newcommand\listfigurename{List of Figures}
\fi
\ifdefined\listtablename
  \renewcommand*\listtablename{List of Tables}
\else
  \newcommand\listtablename{List of Tables}
\fi
\ifdefined\figurename
  \renewcommand*\figurename{Figure}
\else
  \newcommand\figurename{Figure}
\fi
\ifdefined\tablename
  \renewcommand*\tablename{Table}
\else
  \newcommand\tablename{Table}
\fi
}
\@ifpackageloaded{float}{}{\usepackage{float}}
\floatstyle{ruled}
\@ifundefined{c@chapter}{\newfloat{codelisting}{h}{lop}}{\newfloat{codelisting}{h}{lop}[chapter]}
\floatname{codelisting}{Listing}
\newcommand*\listoflistings{\listof{codelisting}{List of Listings}}
\captionsetup{labelsep=colon}
\makeatother
\makeatletter
\makeatother
\makeatletter
\@ifpackageloaded{caption}{}{\usepackage{caption}}
\@ifpackageloaded{subcaption}{}{\usepackage{subcaption}}
\makeatother
\usepackage{bookmark}
\IfFileExists{xurl.sty}{\usepackage{xurl}}{} % add URL line breaks if available
\urlstyle{same}
\hypersetup{
  pdftitle={The Antares Mathematical Framework},
  pdfauthor={Kiran K. Nath},
  pdflang={en-US},
  pdfkeywords={American Options, Spectral Methods, Free Boundary
Problems, Negative Interest Rates, Optimal Stopping},
  colorlinks=true,
  linkcolor={NavyBlue},
  filecolor={Maroon},
  citecolor={NavyBlue},
  urlcolor={NavyBlue},
  pdfcreator={LaTeX via pandoc}}


\title{The Antares Mathematical Framework}
\usepackage{etoolbox}
\makeatletter
\providecommand{\subtitle}[1]{% add subtitle to \maketitle
  \apptocmd{\@title}{\par {\large #1 \par}}{}{}
}
\makeatother
\subtitle{A Spectral Collocation Methodology for American Option Pricing
Under General Interest Rate Conditions}
\author{Kiran K. Nath}
\date{2025-06-28}
\begin{document}
\maketitle
\begin{abstract}
This paper presents a comprehensive mathematical framework for the
valuation of American-style derivative securities that unifies the
treatment of traditional single-boundary configurations with the complex
double-boundary topologies emerging under negative interest rate
environments. The Antares methodology transforms the classical
free-boundary problem into a system of non-linear integral equations
through sophisticated boundary function transformations and spectral
collocation techniques. By employing Chebyshev polynomial interpolation
on carefully regularized boundary representations, the framework
achieves exponential convergence rates while maintaining mathematical
rigor across all interest rate regimes. The key mathematical innovation
lies in the development of decoupled iteration schemes for the
double-boundary case, enabling independent computation of multiple
exercise boundaries through separate fixed-point systems. This
methodology provides both theoretical foundation and practical
computational framework for derivatives pricing under the full spectrum
of modern market conditions, with particular emphasis on the
mathematical elegance of the spectral approach and its convergence
properties.
\end{abstract}


\setstretch{1.5}
\section{Introduction}\label{introduction}

The mathematical valuation of American-style derivative securities
represents one of the most challenging problems in quantitative finance,
fundamentally distinguished from its European counterpart by the
embedded optimal stopping feature that creates a coupled optimization
and valuation problem of considerable analytical complexity. The
essential mathematical difficulty arises from the necessity to
simultaneously determine both the option value function and the optimal
exercise strategy, manifested as a free boundary that separates the
continuation region from the exercise region in the underlying asset
price space.

Traditional numerical approaches to this problem, including binomial and
trinomial lattice methods, finite difference schemes, and Monte Carlo
techniques, typically exhibit algebraic convergence rates that demand
substantial computational resources to achieve precision suitable for
practical applications. The fundamental limitation of these methods
stems from their discrete approximation of continuous processes and
their inability to exploit the inherent smoothness properties of the
value function away from the exercise boundary.

The emergence of negative interest rate regimes in major global
financial markets has introduced additional mathematical complexities
that challenge the foundational assumptions of classical option pricing
theory. Under certain configurations of negative interest rates and
dividend yields, the optimal exercise region can exhibit a qualitatively
different topology characterized by two distinct boundaries rather than
the single boundary assumed in traditional models. This phenomenon,
rigorously analyzed in recent mathematical finance literature, creates
what is termed a ``double continuation region'' where the optimal
exercise strategy involves exercising the option only when the
underlying asset price falls within a specific interval bounded by two
time-dependent functions.

The Antares mathematical framework addresses these challenges through a
unified analytical architecture that extends spectral collocation
methods to encompass both traditional single-boundary configurations and
the more complex double-boundary topologies. The methodology transforms
the free boundary problem into a system of non-linear integral equations
and employs sophisticated mathematical techniques including Chebyshev
polynomial interpolation, high-order quadrature rules, and accelerated
fixed-point iterations to achieve spectral convergence rates.

A particularly significant mathematical contribution of this work lies
in the development of decoupled iteration schemes for the
double-boundary case. Rather than solving a coupled system of equations
for both boundaries simultaneously, the mathematical structure of the
problem allows for the independent computation of each boundary through
separate fixed-point systems. This decoupling not only preserves the
computational efficiency of the single-boundary case but also enhances
numerical stability and provides deeper mathematical insight into the
structure of the optimal stopping problem.

The framework incorporates advanced mathematical transformations at
multiple levels of the analytical hierarchy. The temporal domain
undergoes a square-root transformation that concentrates analytical
effort near the option expiration date where boundary behavior exhibits
the greatest mathematical complexity. The boundary functions themselves
are subjected to logarithmic and power transformations that convert
highly non-linear functions with unbounded derivatives into nearly
linear functions amenable to low-order polynomial approximation. The
integral operators are transformed to eliminate weak singularities that
would otherwise degrade the convergence properties of the spectral
method.

This comprehensive mathematical methodology enables the construction of
a rigorous analytical framework that maintains exceptional accuracy
across the full range of market conditions while providing mathematical
guarantees on convergence behavior and error bounds. The mathematical
rigor of the approach ensures robust performance across extreme
parameter ranges while the spectral convergence properties enable
applications requiring the highest levels of precision.

\section{Mathematical Framework and Problem
Formulation}\label{mathematical-framework-and-problem-formulation}

\subsection{Stochastic Process
Foundation}\label{stochastic-process-foundation}

The mathematical foundation of the Antares framework rests upon the
assumption that the underlying asset price follows a geometric Brownian
motion under the risk-neutral probability measure \(\mathbb{Q}\). This
fundamental modeling choice, while representing a considerable
idealization of actual market dynamics, provides the mathematical
tractability necessary for developing rigorous analytical methods while
maintaining sufficient economic realism for practical applications.

Under the risk-neutral measure, the asset price process
\(\{S(t)\}_{t \geq 0}\) satisfies the stochastic differential equation:

\[\frac{dS(t)}{S(t)} = (r-q)dt + \sigma dW(t) \tag{2.1}\]

where \(r \in \mathbb{R}\) represents the risk-free interest rate,
\(q \in \mathbb{R}\) denotes the continuous dividend yield,
\(\sigma > 0\) characterizes the volatility parameter, and
\(\{W(t)\}_{t \geq 0}\) denotes a standard Wiener process under
\(\mathbb{Q}\). The drift coefficient \((r-q)\) emerges naturally from
the risk-neutralization process and represents the excess return of the
asset over its dividend yield, appropriately adjusted through the change
of measure.

The explicit solution to the stochastic differential equation (2.1)
takes the form:

\[S(t) = S(0) \exp\left(\left(r-q-\frac{1}{2}\sigma^2\right)t + \sigma W(t)\right) \tag{2.2}\]

This representation reveals that the logarithm of the asset price
follows a Brownian motion with drift, leading to the lognormal
distribution for future asset prices. Specifically, for any \(t > 0\),
the conditional distribution of \(S(t)\) given \(S(0) = s\) satisfies:

\[\ln S(t) \mid S(0) = s \sim \mathcal{N}\left(\ln s + \left(r-q-\frac{1}{2}\sigma^2\right)t, \sigma^2 t\right) \tag{2.3}\]

The assumption of constant parameters throughout the option's lifetime
enables the exploitation of time-homogeneity in the underlying process.
This property allows the option value at any time \(t\) for a contract
maturing at time \(T\) to be expressed as a function of the time to
maturity \(\tau \triangleq T-t\) and the current asset price \(S\),
independent of the absolute time \(t\). This dimensional reduction from
a two-dimensional space-time problem to a one-dimensional
time-to-maturity problem significantly simplifies both the mathematical
analysis and the development of numerical algorithms.

\subsection{The Optimal Stopping
Formulation}\label{the-optimal-stopping-formulation}

The mathematical characterization of American option pricing leads
naturally to an optimal stopping problem. For an American put option
with strike price \(K > 0\) and maturity \(T\), the value function at
time \(t\) with underlying asset price \(S(t) = s\) is given by:

\[V(\tau,s) = \sup_{\nu \in \mathcal{T}_{0,\tau}} \mathbb{E}^{\mathbb{Q}}\left[e^{-r\nu}(K-S(\nu))^+ \mid S(0) = s\right] \tag{2.4}\]

where \(\tau = T-t\) represents the time to maturity,
\(\mathcal{T}_{0,\tau}\) denotes the set of all stopping times taking
values in \([0,\tau]\), and the supremum is taken over all admissible
exercise strategies.

The mathematical theory of optimal stopping for Markov processes
guarantees the existence of an optimal stopping time \(\nu^*\) that
achieves the supremum in equation (2.4). Furthermore, this optimal
stopping time can be characterized through a time-dependent exercise
boundary function \(B: [0,\tau] \to \mathbb{R}_+\) such that:

\[\nu^* = \inf\{t \in [0,\tau] : S(t) \leq B(\tau-t)\} \tag{2.5}\]

The exercise boundary \(B(\tau)\) divides the state space into two
regions: the exercise region
\(\mathcal{E}(\tau) = \{s : s \leq B(\tau)\}\) and the continuation
region \(\mathcal{C}(\tau) = \{s : s > B(\tau)\}\).

\subsection{Variational Inequality
Characterization}\label{variational-inequality-characterization}

The value function \(V(\tau,s)\) satisfies a variational inequality that
captures the mathematical essence of the optimal stopping problem. In
the continuation region, the value function must satisfy the
Black-Scholes partial differential equation, while in the exercise
region, the option value equals the intrinsic value. This leads to the
variational inequality:

\[\max\left\{(K-s)^+, \frac{\partial V}{\partial \tau} + \mathcal{L}V\right\} = 0 \tag{2.6}\]

where \(\mathcal{L}\) denotes the infinitesimal generator of the asset
price process:

\[\mathcal{L}V = \frac{1}{2}\sigma^2 s^2 \frac{\partial^2 V}{\partial s^2} + (r-q)s\frac{\partial V}{\partial s} - rV \tag{2.7}\]

The variational inequality (2.6) can be equivalently expressed as the
complementarity system:

\[\begin{aligned}
\frac{\partial V}{\partial \tau} + \mathcal{L}V &\geq 0 \\
V(\tau,s) - (K-s)^+ &\geq 0 \\
\left(\frac{\partial V}{\partial \tau} + \mathcal{L}V\right)(V(\tau,s) - (K-s)^+) &= 0
\end{aligned} \tag{2.8}\]

\subsection{Boundary Conditions and
Regularity}\label{boundary-conditions-and-regularity}

The exercise boundary \(B(\tau)\) must satisfy fundamental conditions
that ensure the optimality of the stopping strategy. The value-matching
condition requires continuity of the value function across the exercise
boundary:

\[V(\tau, B(\tau)) = K - B(\tau) \tag{2.9}\]

The smooth-pasting condition, also known as the high-contact condition,
ensures that the first derivative of the value function is continuous
across the boundary:

\[\frac{\partial V}{\partial s}(\tau, B(\tau)) = -1 \tag{2.10}\]

These conditions, when combined with the variational inequality (2.6),
uniquely determine both the value function \(V(\tau,s)\) and the
exercise boundary \(B(\tau)\).

The mathematical analysis of the boundary function reveals several
important regularity properties. For \(\tau > 0\), the boundary function
\(B(\tau)\) is infinitely differentiable, though all derivatives become
unbounded as \(\tau \to 0^+\). The asymptotic behavior near expiration
depends critically on the relationship between \(r\) and \(q\):

\[\lim_{\tau \to 0^+} B(\tau) = \begin{cases}
K & \text{if } r \geq q \\
K \cdot \frac{r}{q} & \text{if } r < q
\end{cases} \tag{2.11}\]

For large values of \(\tau\), the boundary function converges to the
perpetual American option boundary:

\[\lim_{\tau \to \infty} B(\tau) = K \frac{\lambda_-}{\lambda_- - 1} \tag{2.12}\]

where \(\lambda_-\) represents the negative root of the characteristic
equation:

\[\frac{1}{2}\sigma^2 \lambda^2 + \left(r-q-\frac{1}{2}\sigma^2\right)\lambda - r = 0 \tag{2.13}\]

\section{The Integral Equation
Methodology}\label{the-integral-equation-methodology}

\subsection{Derivation of the Fundamental Integral
Representation}\label{derivation-of-the-fundamental-integral-representation}

The transformation of the American option pricing problem from a partial
differential equation with free boundaries to an integral equation
represents a profound mathematical advancement that forms the
cornerstone of the Antares methodology. This transformation, originally
conceived by Kim and subsequently refined through extensive mathematical
development, provides both computational advantages and deeper
analytical insight into the structure of the early exercise premium.

The derivation begins with the fundamental observation that the American
option value can be decomposed as:

\[V(\tau,s) = v(\tau,s) + \mathcal{P}(\tau,s) \tag{3.1}\]

where \(v(\tau,s)\) represents the corresponding European option price
and \(\mathcal{P}(\tau,s)\) denotes the early exercise premium that
quantifies the additional value provided by the flexibility to exercise
before expiration.

The European put option price is given by the Black-Scholes formula:

\[v(\tau,s) = K e^{-r\tau}\Phi(-d_-(\tau,s/K)) - s e^{-q\tau}\Phi(-d_+(\tau,s/K)) \tag{3.2}\]

where \(\Phi(\cdot)\) denotes the cumulative standard normal
distribution function and the auxiliary functions are defined as:

\[d_{\pm}(\tau,z) \triangleq \frac{\ln(z) + (r-q)\tau \pm \frac{1}{2}\sigma^2\tau}{\sigma\sqrt{\tau}} \tag{3.3}\]

The rigorous derivation of the integral representation employs Itô's
lemma applied to the discounted option value process. Consider the
process \(H(t) = e^{-rt}V(T-t, S(t))\) where \(V\) denotes the American
option value function. Through careful application of the stochastic
calculus, this process can be shown to satisfy:

\[H(\tau) - H(0) = \int_0^\tau e^{-ru} \sigma S(u) \frac{\partial V}{\partial s}(T-u, S(u)) dW(u) + \int_0^\tau e^{-ru} \mathcal{I}(u) du \tag{3.4}\]

where \(\mathcal{I}(u)\) represents the integrand that captures the
early exercise contribution.

Taking expectations and utilizing the martingale property of the
stochastic integral, the early exercise premium can be expressed as:

\[\mathcal{P}(\tau,s) = \int_0^\tau \mathbb{E}^{\mathbb{Q}}\left[e^{-ru} \mathbbm{1}_{\{S(u) \leq B(\tau-u)\}} (rK - qS(u)) \mid S(0) = s\right] du \tag{3.5}\]

The indicator function \(\mathbbm{1}_{\{S(u) \leq B(\tau-u)\}}\)
restricts the integration to periods when early exercise is optimal,
while the term \((rK - qS(u))\) represents the instantaneous net cash
flow benefit from early exercise.

\subsection{Complete Integral Equation
Formulation}\label{complete-integral-equation-formulation}

Utilizing the lognormal distribution properties of the underlying asset
price process, the expectation in equation (3.5) can be evaluated
explicitly, yielding the complete integral representation:

\[V(\tau,s) = v(\tau,s) + \int_0^\tau rK e^{-r(\tau-u)}\Phi(-d_-(\tau-u,s/B(u))) du - \int_0^\tau qs e^{-q(\tau-u)}\Phi(-d_+(\tau-u,s/B(u))) du \tag{3.6}\]

The mathematical elegance of this formulation lies in its economic
interpretation. The first integral term represents the present value of
the interest earned on the strike price \(K\) during periods when early
exercise is optimal, weighted by the risk-neutral probability that the
asset price falls below the exercise boundary. The second integral term
captures the present value of the dividend yield foregone on the
underlying asset position, similarly weighted by the exercise
probability.

The cumulative normal distribution functions
\(\Phi(-d_{\pm}(\tau-u,s/B(u)))\) arise naturally from the lognormal
distribution of future asset prices and represent the risk-neutral
probabilities of early exercise at future times, conditional on the
current asset price \(s\).

\subsection{Boundary Integral
Equations}\label{boundary-integral-equations}

To determine the exercise boundary \(B(\tau)\), the integral equation
(3.6) must be combined with the boundary conditions derived from the
optimal stopping theory. Applying the value-matching condition (2.9)
yields:

\[K - B(\tau) = v(\tau, B(\tau)) + \int_0^\tau rK e^{-r(\tau-u)}\Phi(-d_-(\tau-u,B(\tau)/B(u))) du - \int_0^\tau qB(\tau) e^{-q(\tau-u)}\Phi(-d_+(\tau-u,B(\tau)/B(u))) du \tag{3.7}\]

This nonlinear Volterra-type integral equation for \(B(\tau)\) forms the
foundation of the numerical algorithm. The equation can be rearranged
into the fixed-point form:

\[B(\tau) = K e^{(r-q)\tau} \frac{N(\tau,B)}{D(\tau,B)} \tag{3.8}\]

where the functionals \(N(\tau,B)\) and \(D(\tau,B)\) are defined
through the integral terms and capture the economic forces driving the
exercise decision.

Alternatively, applying the smooth-pasting condition (2.10) and
differentiating the integral equation (3.6) with respect to \(s\) yields
a different but mathematically equivalent boundary equation:

\[-1 = -e^{-q\tau}\Phi(-d_+(τ,B(\tau)/K)) + \int_0^\tau \frac{rK}{B(\tau)} e^{-r(\tau-u)} \frac{\phi(-d_-(\tau-u,B(\tau)/B(u)))}{\sigma\sqrt{\tau-u}} du - \int_0^\tau q e^{-q(\tau-u)} \left[\frac{\phi(-d_+(\tau-u,B(\tau)/B(u)))}{\sigma\sqrt{\tau-u}} + \Phi(-d_+(\tau-u,B(\tau)/B(u)))\right] du \tag{3.9}\]

where \(\phi(\cdot)\) denotes the standard normal probability density
function.

The smooth-pasting formulation (3.9) often exhibits superior numerical
properties due to the symmetry that can be restored between the integral
and non-integral terms through the mathematical identity:

\[\frac{K e^{-r\tau} \phi(-d_-(\tau,B(\tau)/K))}{\sigma\sqrt{\tau}} = \frac{B(\tau) e^{-q\tau} \phi(-d_+(\tau,B(\tau)/K))}{\sigma\sqrt{\tau}} \tag{3.10}\]

This symmetry property proves crucial for the stability and convergence
of the numerical algorithms developed in subsequent sections.

\section{Double Boundary Phenomena Under Negative Interest
Rates}\label{double-boundary-phenomena-under-negative-interest-rates}

\subsection{Mathematical Characterization of the Double Boundary
Regime}\label{mathematical-characterization-of-the-double-boundary-regime}

The emergence of negative interest rate environments has revealed a
mathematically fascinating phenomenon that fundamentally challenges
classical option pricing assumptions. Under specific combinations of
negative interest rates and dividend yields, the optimal exercise region
for American options assumes a qualitatively different topology
characterized by two distinct boundaries rather than the single boundary
traditionally assumed.

For American put options, the double-boundary configuration arises
precisely when the parameters satisfy \(q < r < 0\). The mathematical
foundation for this phenomenon can be understood through the
short-maturity analysis of the exercise premium. In the limit as
\(\tau \to 0^+\), the decision to exercise depends on the sign of the
net carry \((rK - qs)\) for asset prices near the strike.

\textbf{Theorem 4.1} (Double Boundary Existence): \emph{Consider an
American put option with parameters satisfying \(q < r < 0\). There
exists a critical volatility \(\sigma^* = |\sqrt{-2r} - \sqrt{-2q}|\)
such that for \(\sigma \leq \sigma^*\), the optimal exercise region
consists of the interval \([Y(\tau), B(\tau)]\) where both boundaries
are well-defined for all \(\tau \geq 0\).}

The proof of this theorem relies on the asymptotic analysis of the
characteristic equation roots and the behavior of the perpetual option
boundaries. The mathematical conditions ensure that both boundaries
exist and remain finite throughout the option's lifetime.

As \(\tau \to 0^+\), the boundaries approach the limiting values:

\[\lim_{\tau \to 0^+} B(\tau) = K \text{ and } \lim_{\tau \to 0^+} Y(\tau) = K \frac{r}{q} \tag{4.1}\]

Since \(q < r < 0\), the ratio \(r/q\) satisfies \(0 < r/q < 1\),
establishing that \(Y(0^+) < B(0^+) = K\), confirming the existence of a
non-trivial exercise interval near expiration.

\subsection{Long-Term Asymptotic
Analysis}\label{long-term-asymptotic-analysis}

The long-term behavior of the double boundaries depends critically on
the volatility relative to the critical threshold \(\sigma^*\). When
\(\sigma < \sigma^*\), both boundaries converge to finite limits as
\(\tau \to \infty\):

\[\lim_{\tau \to \infty} B(\tau) = K \frac{\lambda_+}{\lambda_+ - 1} \text{ and } \lim_{\tau \to \infty} Y(\tau) = K \frac{\lambda_-}{\lambda_- - 1} \tag{4.2}\]

where \(\lambda_{\pm}\) represent the roots of the modified
characteristic equation:

\[\lambda_{\pm} = \frac{-\mu \pm \sqrt{\mu^2 + 2r\sigma^2}}{\sigma^2} \text{ with } \mu = r - q - \frac{\sigma^2}{2} \tag{4.3}\]

When \(\sigma > \sigma^*\), the boundaries intersect at a finite time
\(\tau^*\), and for \(\tau > \tau^*\), the exercise region disappears
entirely, rendering the American option equivalent to its European
counterpart.

\textbf{Proposition 4.1} (Boundary Intersection): \emph{For parameters
satisfying \(q < r < 0\) and \(\sigma > \sigma^*\), there exists a
unique time \(\tau^* < \infty\) such that \(B(\tau^*) = Y(\tau^*)\) and
the exercise region is empty for all \(\tau > \tau^*\).}

The intersection time \(\tau^*\) cannot be determined analytically in
general but can be computed numerically through root-finding algorithms
applied to the equation \(B(\tau^*) = Y(\tau^*)\).

\subsection{Modified Integral Equation
Formulation}\label{modified-integral-equation-formulation}

The presence of two exercise boundaries necessitates a fundamental
modification of the integral equation formulation. The early exercise
premium must now account for the finite exercise region rather than the
semi-infinite region of the single-boundary case.

The modified integral representation takes the form:

\[V(\tau,s) = v(\tau,s) + \int_0^{\min(\tau,\tau^*)} \mathbb{E}^{\mathbb{Q}}\left[e^{-ru} \mathbbm{1}_{\{Y(\tau-u) \leq S(u) \leq B(\tau-u)\}} (rK - qS(u)) \mid S(0) = s\right] du \tag{4.4}\]

The indicator function now restricts the integration to the finite
interval \([Y(\tau-u), B(\tau-u)]\), reflecting the bounded nature of
the exercise region.

Evaluating the expectation yields the complete double-boundary integral
equation:

\[V(\tau,s) = v(\tau,s) + \int_0^{\min(\tau,\tau^*)} rK e^{-r(\tau-u)}[\Phi(-d_-(\tau-u,s/B(u))) - \Phi(-d_-(\tau-u,s/Y(u)))] du - \int_0^{\min(\tau,\tau^*)} qs e^{-q(\tau-u)}[\Phi(-d_+(\tau-u,s/B(u))) - \Phi(-d_+(\tau-u,s/Y(u)))] du \tag{4.5}\]

\subsection{Decoupling of the Boundary
Systems}\label{decoupling-of-the-boundary-systems}

A remarkable mathematical property of the double-boundary problem is
that the two boundaries can be computed independently despite their
apparent coupling in the integral equation (4.5). This decoupling arises
from the fundamental principle that for asset prices outside the
exercise interval, the option value depends only on the boundary that
would be encountered first by the diffusion process.

\textbf{Theorem 4.2} (Boundary Decoupling): \emph{The upper boundary
\(B(\tau)\) satisfies the single-boundary integral equation for all
asset prices \(s \geq B(\tau)\), while the lower boundary \(Y(\tau)\)
satisfies a modified integral equation for all asset prices
\(s \leq Y(\tau)\).}

For the upper boundary \(B(\tau)\), applying the value-matching
condition at \(s = B(\tau)\) yields:

\[K - B(\tau) = v(\tau, B(\tau)) + \int_0^{\min(\tau,\tau^*)} rK e^{-r(\tau-u)}\Phi(-d_-(\tau-u,B(\tau)/B(u))) du - \int_0^{\min(\tau,\tau^*)} qB(\tau) e^{-q(\tau-u)}\Phi(-d_+(\tau-u,B(\tau)/B(u))) du \tag{4.6}\]

For the lower boundary \(Y(\tau)\), applying the smooth-pasting
condition at \(s = Y(\tau)\) and utilizing the mathematical structure of
the problem yields:

\[-1 = -e^{-q\tau}\Phi(-d_+(τ,Y(\tau)/K)) + \int_0^{\min(\tau,\tau^*)} \frac{rK}{Y(\tau)} e^{-r(\tau-u)} \frac{\phi(-d_-(\tau-u,Y(\tau)/Y(u)))}{\sigma\sqrt{\tau-u}} du - \int_0^{\min(\tau,\tau^*)} q e^{-q(\tau-u)} \left[\frac{\phi(-d_+(\tau-u,Y(\tau)/Y(u)))}{\sigma\sqrt{\tau-u}} + \Phi(-d_+(\tau-u,Y(\tau)/Y(u)))\right] du \tag{4.7}\]

This decoupling enables the development of separate fixed-point
iteration schemes for each boundary, dramatically simplifying the
computational problem while maintaining mathematical rigor.

\section{The Spectral Collocation
Methodology}\label{the-spectral-collocation-methodology}

\subsection{Mathematical Foundation of Spectral
Approximation}\label{mathematical-foundation-of-spectral-approximation}

The spectral collocation methodology employed in the Antares framework
represents a sophisticated application of global polynomial
approximation theory to the solution of nonlinear integral equations
arising in American option pricing. The mathematical foundation rests
upon the exceptional approximation properties of orthogonal polynomials
and their ability to achieve exponential convergence rates for
sufficiently smooth functions.

The fundamental principle of spectral methods lies in the representation
of the unknown function through global basis functions that possess
optimal approximation properties. For the exercise boundary function
\(B(\tau)\), the spectral approximation takes the form:

\[B_N(\tau) = \sum_{k=0}^{N} c_k \mathcal{B}_k(\tau) \tag{5.1}\]

where \(\{\mathcal{B}_k(\tau)\}_{k=0}^{N}\) represents a system of
orthogonal basis functions and \(\{c_k\}_{k=0}^{N}\) denotes the
expansion coefficients to be determined.

The choice of basis functions proves crucial for the success of the
spectral method. Chebyshev polynomials of the first kind provide optimal
approximation properties for functions defined on bounded intervals and
are characterized by the three-term recurrence relation:

\[T_0(x) = 1, \quad T_1(x) = x, \quad T_{k+1}(x) = 2xT_k(x) - T_{k-1}(x) \tag{5.2}\]

with the orthogonality relation:

\[\int_{-1}^{1} T_j(x) T_k(x) \frac{dx}{\sqrt{1-x^2}} = \begin{cases}
\pi & \text{if } j = k = 0 \\
\pi/2 & \text{if } j = k \neq 0 \\
0 & \text{if } j \neq k
\end{cases} \tag{5.3}\]

\subsection{Domain Transformation and
Regularization}\label{domain-transformation-and-regularization}

The raw exercise boundary function \(B(\tau)\) exhibits singular
behavior near \(\tau = 0\) that precludes direct polynomial
approximation. The boundary possesses unbounded derivatives of all
orders at the origin, making standard polynomial approximation
inefficient and potentially unstable.

The Antares methodology addresses this challenge through a sophisticated
sequence of mathematical transformations designed to regularize the
boundary function and render it amenable to spectral approximation. The
transformation proceeds in several carefully designed stages.

\textbf{Stage 1: Temporal Domain Transformation}

The temporal domain \([0, \tau_{\max}]\) is first transformed to
concentrate computational effort near the critical region around
expiration. The square-root transformation:

\[\xi = \sqrt{\tau/\tau_{\max}} \tag{5.4}\]

maps the time-to-maturity variable to the interval \([0,1]\) while
providing increased resolution near \(\tau = 0\) where the boundary
exhibits the most complex behavior.

\textbf{Stage 2: Boundary Normalization}

The boundary function is normalized by its short-maturity limiting value
to remove dependence on the strike price and account for the
discontinuity that occurs when \(r < q\):

\[\widetilde{B}(\tau) = \frac{B(\tau)}{X} \text{ where } X = K \min(1, r/q) \tag{5.5}\]

\textbf{Stage 3: Logarithmic Transformation}

A logarithmic transformation is applied to linearize the exponential
decay behavior and compress the range of the function:

\[G(\xi) = \ln(\widetilde{B}(\xi^2)) \tag{5.6}\]

\textbf{Stage 4: Variance-Stabilizing Transformation}

The final transformation employs a variance-stabilizing technique that
converts the function to a nearly linear form:

\[H(\xi) = G(\xi)^2 = [\ln(\widetilde{B}(\xi^2))]^2 \tag{5.7}\]

\textbf{Theorem 5.1} (Transformation Regularity): \emph{The composite
transformation \(H(\xi)\) defined by equation (5.7) is infinitely
differentiable on \((0,1]\) and exhibits polynomial growth in its
derivatives, ensuring exponential convergence of the Chebyshev
approximation.}

The proof of this theorem relies on the asymptotic analysis of the
boundary function and demonstrates that the transformed function
\(H(\xi)\) possesses the regularity properties necessary for spectral
convergence.

\subsection{Chebyshev Collocation
Implementation}\label{chebyshev-collocation-implementation}

The collocation method determines the expansion coefficients by
requiring the approximation to satisfy the governing equation at a
discrete set of carefully chosen points. For Chebyshev polynomials, the
optimal choice of collocation points corresponds to the
Chebyshev-Gauss-Lobatto nodes:

\[\xi_j = \cos\left(\frac{j\pi}{N}\right), \quad j = 0, 1, \ldots, N \tag{5.8}\]

These points cluster near the boundaries of the interval \([-1,1]\) and
provide optimal conditioning for the resulting linear system.

The boundary values \(H(\xi_j)\) at the collocation points determine the
Chebyshev expansion coefficients through the discrete cosine transform:

\[a_k = \frac{2}{N} \sum_{j=0}^{N} \frac{H(\xi_j)}{c_k} \cos\left(\frac{jk\pi}{N}\right) \tag{5.9}\]

where \(c_0 = 2\) and \(c_k = 1\) for \(k \geq 1\).

The evaluation of the interpolating polynomial at arbitrary points
employs the numerically stable Clenshaw recurrence algorithm:

\[\begin{aligned}
b_{N+1} &= b_{N+2} = 0 \\
b_k &= a_k + 2\xi b_{k+1} - b_{k+2}, \quad k = N, N-1, \ldots, 1 \\
H(\xi) &= a_0 + \xi b_1 - b_2
\end{aligned} \tag{5.10}\]

\subsection{Convergence Analysis}\label{convergence-analysis}

The convergence properties of the spectral collocation method depend
fundamentally on the smoothness of the transformed boundary function.
For the Antares transformation sequence, the convergence can be
characterized through the following theorem.

\textbf{Theorem 5.2} (Spectral Convergence): \emph{Let \(H(\xi)\) denote
the transformed boundary function defined by equation (5.7), and let
\(H_N(\xi)\) denote its \(N\)-point Chebyshev interpolant. If \(H\) is
analytic in a region containing \([-1,1]\), then the approximation error
satisfies:}

\[\|H - H_N\|_{\infty} \leq C \rho^{-N} \tag{5.11}\]

\emph{where \(C\) is a constant depending on the function and
\(\rho > 1\) is determined by the location of the nearest singularity in
the complex plane.}

The exponential convergence rate characterized by equation (5.11)
represents a dramatic improvement over the algebraic convergence rates
\(O(N^{-p})\) achieved by finite difference methods. This superior
convergence enables the achievement of high accuracy with relatively few
degrees of freedom, typically requiring only 5-8 collocation points for
most practical applications.

\subsection{Integration of Singular
Kernels}\label{integration-of-singular-kernels}

The integral operators appearing in the boundary equations often contain
weak singularities that require careful mathematical treatment to
maintain the accuracy of the spectral method. The smooth-pasting
formulation contains integrals of the form:

\[\mathcal{I}(\tau) = \int_0^\tau \frac{f(u)}{\sqrt{\tau-u}} du \tag{5.12}\]

where the factor \((\tau-u)^{-1/2}\) creates a weak singularity at
\(u = \tau\).

The Antares methodology eliminates this singularity through the
analytical transformation \(z = \sqrt{\tau-u}\), which yields
\(du = -2z \, dz\) and converts the integral to:

\[\mathcal{I}(\tau) = 2\int_0^{\sqrt{\tau}} f(\tau-z^2) z \, dz \tag{5.13}\]

This transformation completely removes the singularity and produces a
smooth integrand amenable to high-order quadrature rules. The additional
factor of \(z\) ensures that the transformed integral remains
well-behaved at \(z = 0\).

For computational efficiency, the integration domain is further
normalized to the standard interval \([-1,1]\) through:

\[y = -1 + \frac{2z}{\sqrt{\tau}} = -1 + 2\sqrt{\frac{\tau-u}{\tau}} \tag{5.14}\]

This normalization enables the use of standard Gaussian quadrature
weights and nodes, eliminating the need to recompute quadrature
parameters for each value of \(\tau\).

\section{Fixed-Point Iteration and Convergence
Acceleration}\label{fixed-point-iteration-and-convergence-acceleration}

\subsection{Mathematical Structure of the Fixed-Point
Problem}\label{mathematical-structure-of-the-fixed-point-problem}

The boundary integral equations derived in the previous sections can be
cast in the abstract fixed-point form \(B = \mathcal{F}(B)\) where
\(\mathcal{F}\) represents a nonlinear operator mapping boundary
functions to boundary functions. The mathematical properties of this
operator determine the convergence behavior of iterative solution
methods.

For the single-boundary case, the fixed-point operator takes the form:

\[\mathcal{F}(B)(\tau) = K e^{(r-q)\tau} \frac{N(\tau,B)}{D(\tau,B)} \tag{6.1}\]

where the functionals \(N(\tau,B)\) and \(D(\tau,B)\) are defined
through the integral terms in the boundary equation.

\textbf{Theorem 6.1} (Contraction Mapping): \emph{Under suitable
regularity conditions on the parameters \((r,q,\sigma)\) and for
boundary functions in an appropriate function space, the operator
\(\mathcal{F}\) defined by equation (6.1) is a contraction mapping with
Lipschitz constant \(L < 1\).}

The proof of this theorem employs the analysis of the Gâteaux derivative
of the operator \(\mathcal{F}\) and demonstrates that small
perturbations in the boundary function lead to proportionally smaller
perturbations in the image, ensuring convergence of the fixed-point
iteration.

\subsection{Accelerated Iteration
Schemes}\label{accelerated-iteration-schemes}

While the basic fixed-point iteration
\(B^{(k+1)} = \mathcal{F}(B^{(k)})\) guarantees convergence under the
contraction mapping property, the convergence rate is typically linear
and may require numerous iterations to achieve high precision. The
Antares methodology employs sophisticated acceleration techniques to
achieve quadratic convergence rates.

The Jacobi-Newton acceleration scheme incorporates information about the
functional derivative of the operator \(\mathcal{F}\) to achieve
superlinear convergence. The iteration takes the form:

\[B^{(k+1)}(\tau) = B^{(k)}(\tau) + \eta \frac{B^{(k)}(\tau) - \mathcal{F}(B^{(k)})(\tau)}{\mathcal{F}'(B^{(k)})(\tau) - 1} \tag{6.2}\]

where \(\mathcal{F}'(B)(\tau)\) denotes the Gâteaux derivative of the
operator with respect to boundary perturbations at the point \(\tau\),
and \(\eta \in (0,1]\) represents a relaxation parameter.

The computation of the functional derivative requires evaluation of
additional integral expressions but typically reduces the required
number of iterations from dozens to fewer than five, providing
substantial overall computational savings.

\textbf{Lemma 6.1} (Gâteaux Derivative): \emph{For both single-boundary
fixed-point systems, the Gâteaux derivative of the operator
\(\mathcal{F}\) with respect to proportional boundary perturbations
\(\ln B(\tau) \to \ln B(\tau) + \omega g(\tau)\) satisfies:}

\[\frac{\partial \mathcal{F}}{\partial \omega}\bigg|_{\omega=0} = \frac{K e^{(r-q)\tau}}{D(\tau,B)} \int_0^\tau e^{ru} \frac{r-q}{B(u)/K} \psi(\tau-u, B(\tau)/B(u)) \frac{\phi(d_-(\tau-u,B(\tau)/B(u)))}{\sigma\sqrt{\tau-u}} (g(\tau) - g(u)) du \tag{6.3}\]

\emph{where \(\psi\) is a system-dependent weighting function.}

\subsection{Anderson Acceleration}\label{anderson-acceleration}

For boundary functions that exhibit slowly converging fixed-point
behavior, the Antares framework incorporates Anderson acceleration, a
multisecant method that combines information from multiple previous
iterates to accelerate convergence.

The Anderson acceleration maintains a history of the most recent \(m\)
iterates and residuals:

\[\mathbf{F}_k = [F_k, F_{k-1}, \ldots, F_{k-m+1}], \quad \mathbf{G}_k = [G_k, G_{k-1}, \ldots, G_{k-m+1}] \tag{6.4}\]

where \(F_j = B^{(j)}\) and \(G_j = \mathcal{F}(B^{(j)}) - B^{(j)}\)
represent the iterates and residuals, respectively.

The accelerated iterate is computed as:

\[B^{(k+1)} = \mathcal{F}\left(\mathbf{F}_k \boldsymbol{\alpha}\right) \tag{6.5}\]

where the weight vector \(\boldsymbol{\alpha}\) solves the least-squares
problem:

\[\boldsymbol{\alpha} = \arg\min_{\boldsymbol{\beta}} \left\|\mathbf{G}_k \boldsymbol{\beta}\right\|_2 \text{ subject to } \mathbf{e}^T \boldsymbol{\beta} = 1 \tag{6.6}\]

This acceleration technique often achieves superlinear convergence for
well-conditioned problems and provides robustness against
ill-conditioning that might affect the Jacobi-Newton method.

\subsection{Double-Boundary Iteration
Schemes}\label{double-boundary-iteration-schemes}

The mathematical decoupling established in Theorem 4.2 enables the
development of separate iteration schemes for the upper and lower
boundaries in the double-boundary case. However, the practical
implementation reveals that different acceleration strategies prove
optimal for each boundary.

For the upper boundary \(B(\tau)\), which typically exhibits more
nonlinear behavior, the full Jacobi-Newton acceleration provides optimal
performance:

\[B^{(k+1)}(\tau) = B^{(k)}(\tau) + \eta \frac{B^{(k)}(\tau) - \mathcal{F}_B(B^{(k)})(\tau)}{\mathcal{F}_B'(B^{(k)})(\tau) - 1} \tag{6.7}\]

For the lower boundary \(Y(\tau)\), which tends to be smoother and more
linear, simplified acceleration schemes or even standard fixed-point
iteration often suffice:

\[Y^{(k+1)}(\tau) = \mathcal{F}_Y(Y^{(k)})(\tau) \tag{6.8}\]

where \(\mathcal{F}_Y\) denotes the fixed-point operator derived from
the smooth-pasting condition for the lower boundary.

\subsection{Convergence Monitoring and Adaptive
Control}\label{convergence-monitoring-and-adaptive-control}

The Antares framework incorporates sophisticated convergence monitoring
that employs multiple criteria to ensure robust termination of the
iterative process. The primary convergence criterion monitors the
maximum relative change in the boundary function:

\[\epsilon_{\text{rel}} = \max_{\tau \in [0,\tau_{\max}]} \left|\frac{B^{(k+1)}(\tau) - B^{(k)}(\tau)}{B^{(k)}(\tau)}\right| \tag{6.9}\]

Secondary criteria detect oscillatory behavior and stagnation:

\[\epsilon_{\text{osc}} = \max_{\tau \in [0,\tau_{\max}]} \left|B^{(k+1)}(\tau) - B^{(k-1)}(\tau)\right| \tag{6.10}\]

\[\epsilon_{\text{stag}} = \max_{j=1,\ldots,5} \left\|\mathbf{B}^{(k)} - \mathbf{B}^{(k-j)}\right\|_{\infty} \tag{6.11}\]

The iteration terminates when either high accuracy is achieved
(\(\epsilon_{\text{rel}} < \epsilon_{\text{tol}}\)) or when the
convergence rate indicates that further iterations are unlikely to
improve the solution significantly.

\section{Mathematical Validation and Error
Analysis}\label{mathematical-validation-and-error-analysis}

\subsection{Theoretical Error Bounds}\label{theoretical-error-bounds}

The mathematical validation of the Antares methodology requires rigorous
analysis of the various sources of approximation error and their
cumulative effect on the final option price computation. The total error
can be decomposed into three primary components: boundary approximation
error, integral approximation error, and iteration truncation error.

\textbf{Boundary Approximation Error}: The spectral approximation of the
transformed boundary function \(H(\xi)\) through its \(N\)-point
Chebyshev interpolant \(H_N(\xi)\) introduces an error that propagates
through the inverse transformation to the original boundary function.
The error bound can be established through the following theorem.

\textbf{Theorem 7.1} (Boundary Error Propagation): \emph{Let \(B(\tau)\)
denote the exact exercise boundary and \(B_N(\tau)\) denote the boundary
reconstructed from the \(N\)-point Chebyshev approximation of the
transformed function \(H(\xi)\). If the transformation sequence
satisfies the regularity conditions of Theorem 5.1, then:}

\[\|B - B_N\|_{\infty} \leq C_1 \|H - H_N\|_{\infty} \leq C_1 C_2 \rho^{-N} \tag{7.1}\]

\emph{where \(C_1\) depends on the transformation Jacobian and \(C_2\)
depends on the smoothness of \(H(\xi)\).}

\textbf{Integral Approximation Error}: The evaluation of the integral
operators through high-order quadrature rules introduces a second source
of approximation error. For Gaussian quadrature with \(L\) nodes applied
to smooth integrands, the error satisfies:

\[\left|\int_0^\tau f(u) du - \sum_{l=1}^{L} w_l f(u_l)\right| \leq C_3 \tau^{2L+1} \max_{u \in [0,\tau]} |f^{(2L)}(u)| \tag{7.2}\]

where \(\{w_l, u_l\}_{l=1}^{L}\) represent the quadrature weights and
nodes.

\textbf{Iteration Truncation Error}: The finite number of fixed-point
iterations introduces a third error component that depends on the
convergence rate of the iteration scheme. For the accelerated methods,
this error typically decreases quadratically:

\[\|B^{(M)} - B^*\|_{\infty} \leq C_4 \lambda^M \tag{7.3}\]

where \(M\) represents the number of iterations, \(B^*\) denotes the
exact fixed-point solution, and \(\lambda < 1\) characterizes the
convergence rate.

\subsection{Option Price Error
Analysis}\label{option-price-error-analysis}

The ultimate objective of the boundary computation is the accurate
evaluation of option prices. The error in the option price depends on
both the accuracy of the boundary approximation and the precision of the
final price integral evaluation.

\textbf{Theorem 7.2} (Option Price Error Bound): \emph{Let \(V(\tau,s)\)
denote the exact American option price and \(V_h(\tau,s)\) denote the
computed approximation using boundary approximation error
\(\|B - B_h\|_{\infty} \leq h\). Then the option price error satisfies:}

\[|V(\tau,s) - V_h(\tau,s)| \leq C_5 h + C_6 \varepsilon_{\text{quad}} \tag{7.4}\]

\emph{where \(\varepsilon_{\text{quad}}\) represents the quadrature
error in the final price integral and the constants \(C_5, C_6\) depend
on the option parameters and the boundary sensitivity.}

The proof of this theorem employs the Lipschitz continuity of the option
price functional with respect to the boundary function and demonstrates
that the price error scales linearly with the boundary approximation
error.

\subsection{Convergence Verification}\label{convergence-verification}

The practical verification of convergence requires comparison with
high-precision benchmark solutions. The Antares framework employs
multiple validation strategies to ensure the reliability of computed
results.

\textbf{Richardson Extrapolation}: Multiple solutions are computed with
different numbers of collocation points \(N_1 < N_2 < N_3\), and
Richardson extrapolation is used to estimate the error:

\[V_{\text{extrap}} = V_{N_3} + \frac{V_{N_3} - V_{N_2}}{(N_3/N_2)^p - 1} \tag{7.5}\]

where \(p\) represents the theoretical convergence order.

\textbf{Method Comparison}: The spectral results are compared with
high-precision finite difference solutions computed on extremely fine
grids. These comparisons consistently demonstrate the superior accuracy
of the spectral approach.

\textbf{Asymptotic Verification}: For limiting cases where analytical
solutions exist (such as perpetual options or European options), the
computed results are compared with the exact formulas to verify the
correctness of the implementation.

\subsection{Greeks Computation and Sensitivity
Analysis}\label{greeks-computation-and-sensitivity-analysis}

The spectral representation of the boundary function enables the
analytical computation of option price sensitivities (Greeks) with
exceptional accuracy. The Greeks can be computed through either direct
differentiation of the price formula or through automatic
differentiation of the boundary computation algorithm.

\textbf{Delta Computation}: The hedge ratio
\(\Delta = \partial V/\partial S\) can be computed analytically from the
integral representation:

\[\Delta(\tau,s) = \frac{\partial v}{\partial s}(\tau,s) + \int_0^\tau \left[rK e^{-r(\tau-u)} + qs e^{-q(\tau-u)}\right] \frac{\phi(d_-(\tau-u,s/B(u)))}{\sigma\sqrt{\tau-u}} \frac{1}{s} du \tag{7.6}\]

\textbf{Gamma Computation}: The convexity
\(\Gamma = \partial^2 V/\partial S^2\) follows from differentiating the
delta expression:

\[\Gamma(\tau,s) = \frac{\partial^2 v}{\partial s^2}(\tau,s) + \int_0^\tau \frac{rK e^{-r(\tau-u)} + qs e^{-q(\tau-u)}}{\sigma\sqrt{\tau-u}} \frac{\phi(d_-(\tau-u,s/B(u)))}{s^2} \left[\frac{d_-(\tau-u,s/B(u))}{\sigma\sqrt{\tau-u}} - 1\right] du \tag{7.7}\]

The spectral accuracy of the boundary representation ensures that the
Greeks maintain their precision throughout the differentiation process,
providing reliable sensitivity measures for risk management
applications.

\section{Extensions and Mathematical
Generalizations}\label{extensions-and-mathematical-generalizations}

\subsection{Time-Dependent Parameter
Extensions}\label{time-dependent-parameter-extensions}

The mathematical framework of the Antares methodology can be extended to
accommodate time-dependent interest rates, dividend yields, and
volatilities through careful modification of the integral equation
formulation. Consider the stochastic differential equation:

\[\frac{dS(t)}{S(t)} = (r(t)-q(t))dt + \sigma(t) dW(t) \tag{8.1}\]

where the parameters are now deterministic functions of time.

The modified integral representation takes the form:

\[V(\tau,s) = v_T(\tau,s) + \int_0^\tau r(T-u)K P(T-\tau,T-u) \Phi(-\tilde{d}_-(u,s/B(u),T-\tau)) du - \int_0^\tau q(T-u)s Q(T-\tau,T-u) \Phi(-\tilde{d}_+(u,s/B(u),T-\tau)) du \tag{8.2}\]

where \(P(t_1,t_2) = \exp(-\int_{t_1}^{t_2} r(u) du)\) and
\(Q(t_1,t_2) = \exp(-\int_{t_1}^{t_2} q(u) du)\) represent the
integrated discount factors, and the modified \(d\)-functions account
for the time-dependent parameters:

\[\tilde{d}_{\pm}(\delta,z,t) = \frac{\ln(z \cdot Q(t,t+\delta)/P(t,t+\delta)) \pm \frac{1}{2}\Sigma(t,t+\delta)}{\sqrt{\Sigma(t,t+\delta)}} \tag{8.3}\]

with \(\Sigma(t_1,t_2) = \int_{t_1}^{t_2} \sigma(u)^2 du\).

\subsection{Jump-Diffusion Extensions}\label{jump-diffusion-extensions}

The incorporation of jump components in the underlying asset price
dynamics requires fundamental modifications to both the stochastic
process specification and the integral equation formulation. Consider
the jump-diffusion process:

\[\frac{dS(t)}{S(t-)} = (r-q-\lambda\kappa)dt + \sigma dW(t) + \int_{\mathbb{R}} z \tilde{N}(dt,dz) \tag{8.4}\]

where \(\tilde{N}(dt,dz)\) represents a compensated Poisson random
measure with intensity \(\lambda\) and jump size distribution
characterized by the Lévy measure \(\nu(dz)\).

The modified integral equation must account for the additional
contribution from jumps that cross the exercise boundary:

\[V(\tau,s) = v_J(\tau,s) + \int_0^\tau \mathbb{E}^{\mathbb{Q}}\left[e^{-ru} \mathbbm{1}_{\{S(u-) \leq B(\tau-u)\}} (rK - qS(u-)) \mid S(0) = s\right] du + \int_0^\tau \lambda \int_{\mathbb{R}} \mathbb{E}^{\mathbb{Q}}\left[e^{-ru} (V(\tau-u,S(u-)e^z) - V(\tau-u,S(u-))) \mathbbm{1}_{\{S(u-) > B(\tau-u), S(u-)e^z \leq B(\tau-u)\}} \mid S(0) = s\right] \nu(dz) du \tag{8.5}\]

where \(v_J(\tau,s)\) denotes the corresponding European option price
under the jump-diffusion model.

\subsection{Stochastic Volatility
Extensions}\label{stochastic-volatility-extensions}

The extension to stochastic volatility models, such as the Heston model,
requires a fundamental increase in dimensionality as the exercise
boundary becomes a function of both time and the volatility state
variable. Consider the Heston system:

\[\begin{aligned}
\frac{dS(t)}{S(t)} &= (r-q)dt + \sqrt{V(t)} dW_1(t) \\
dV(t) &= \kappa(\theta - V(t))dt + \sigma_v \sqrt{V(t)} dW_2(t)
\end{aligned} \tag{8.6}\]

where \(dW_1(t) dW_2(t) = \rho dt\).

The exercise boundary becomes a surface \(B(\tau,v)\) in the
two-dimensional state space, and the integral equation formulation
requires two-dimensional integration over the exercise region. The
spectral collocation method can be extended through tensor product
constructions using bivariate Chebyshev polynomials:

\[B_N(\tau,v) = \sum_{j=0}^{N_1} \sum_{k=0}^{N_2} c_{jk} T_j(\xi_\tau) T_k(\xi_v) \tag{8.7}\]

where \(\xi_\tau\) and \(\xi_v\) represent transformed time and
volatility variables, respectively.

\subsection{Multi-Asset Extensions}\label{multi-asset-extensions}

The extension to multi-asset American options, such as exchange options
or basket options, follows naturally from the single-asset framework.
For a two-asset exchange option with payoff \((S_1 - S_2)^+\), the
exercise boundary becomes a curve in the two-dimensional asset price
space.

The correlated geometric Brownian motion system:

\[\begin{aligned}
\frac{dS_1(t)}{S_1(t)} &= (r-q_1)dt + \sigma_1 dW_1(t) \\
\frac{dS_2(t)}{S_2(t)} &= (r-q_2)dt + \sigma_2 dW_2(t)
\end{aligned} \tag{8.8}\]

with \(dW_1(t) dW_2(t) = \rho dt\) leads to a two-dimensional optimal
stopping problem with exercise boundary \(B(\tau,s_2)\) representing the
critical level of \(S_1\) as a function of time and the level of
\(S_2\).

The spectral representation of the boundary surface employs
two-dimensional Chebyshev expansion:

\[B(\tau,s_2) = \sum_{j=0}^{N_1} \sum_{k=0}^{N_2} c_{jk} T_j(\xi_\tau) T_k(\xi_{s_2}) \tag{8.9}\]

where appropriate transformations are applied to both the temporal and
spatial dimensions.

\section{Conclusion}\label{conclusion}

The Antares mathematical framework represents a comprehensive analytical
architecture that fundamentally advances the theoretical and
computational treatment of American option pricing across the full
spectrum of interest rate conditions. Through the innovative combination
of optimal stopping theory, integral equation transformations, and
spectral collocation techniques, the methodology achieves unprecedented
mathematical rigor while maintaining exceptional computational
efficiency.

The mathematical elegance of the approach lies in its unified treatment
of both traditional single-boundary configurations and the complex
double-boundary topologies that emerge under negative interest rate
conditions. The development of decoupled iteration schemes for the
double-boundary case represents a particularly significant mathematical
contribution, enabling the independent computation of multiple exercise
boundaries while preserving the convergence properties of the
single-boundary framework.

The transformation sequence that converts singular, highly nonlinear
boundary functions into smooth, nearly linear functions amenable to
spectral approximation demonstrates the power of appropriate
mathematical preprocessing in numerical algorithm design. The resulting
exponential convergence rates represent a qualitative improvement over
traditional algebraic methods and enable the achievement of machine
precision accuracy with minimal computational effort.

The rigorous mathematical analysis of convergence properties, error
bounds, and stability characteristics provides theoretical guarantees
that complement the exceptional empirical performance. The framework's
ability to maintain mathematical rigor across challenging parameter
regimes, including those that arise in modern negative interest rate
environments, ensures its applicability to the full range of
contemporary financial market conditions.

The extensibility of the mathematical framework to accommodate
time-dependent parameters, alternative underlying processes, and
multi-dimensional problems demonstrates the fundamental soundness of the
analytical approach. The spectral methodology provides a robust
foundation for addressing increasingly complex derivative valuation
challenges while maintaining the mathematical precision required for
critical financial applications.

The Antares framework thus establishes a new standard for the
mathematical treatment of American option pricing problems, combining
theoretical rigor with computational excellence in a unified analytical
architecture. The methodology's emphasis on mathematical elegance,
provable convergence properties, and practical efficiency positions it
as a foundational contribution to the field of computational finance
that will enable both current applications and future theoretical
developments.

\begin{center}\rule{0.5\linewidth}{0.5pt}\end{center}


\printbibliography



\end{document}
